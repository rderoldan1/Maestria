
%% bare_conf.tex
%% V1.4
%% 2012/12/27
%% by Michael Shell
%% See:
%% http://www.michaelshell.org/
%% for current contact information.
%%
%% This is a skeleton file demonstrating the use of IEEEtran.cls
%% (requires IEEEtran.cls version 1.8 or later) with an IEEE conference paper.
%%
%% Support sites:
%% http://www.michaelshell.org/tex/ieeetran/
%% http://www.ctan.org/tex-archive/macros/latex/contrib/IEEEtran/
%% and
%% http://www.ieee.org/

%%*************************************************************************
%% Legal Notice:
%% This code is offered as-is without any warranty either expressed or
%% implied; without even the implied warranty of MERCHANTABILITY or
%% FITNESS FOR A PARTICULAR PURPOSE! 
%% User assumes all risk.
%% In no event shall IEEE or any contributor to this code be liable for
%% any damages or losses, including, but not limited to, incidental,
%% consequential, or any other damages, resulting from the use or misuse
%% of any information contained here.
%%
%% All comments are the opinions of their respective authors and are not
%% necessarily endorsed by the IEEE.
%%
%% This work is distributed under the LaTeX Project Public License (LPPL)
%% ( http://www.latex-project.org/ ) version 1.3, and may be freely used,
%% distributed and modified. A copy of the LPPL, version 1.3, is included
%% in the base LaTeX documentation of all distributions of LaTeX released
%% 2003/12/01 or later.
%% Retain all contribution notices and credits.
%% ** Modified files should be clearly indicated as such, including  **
%% ** renaming them and changing author support contact information. **
%%
%% File list of work: IEEEtran.cls, IEEEtran_HOWTO.pdf, bare_adv.tex,
%%                    bare_conf.tex, bare_jrnl.tex, bare_jrnl_compsoc.tex,
%%                    bare_jrnl_transmag.tex
%%*************************************************************************

% *** Authors should verify (and, if needed, correct) their LaTeX system  ***
% *** with the testflow diagnostic prior to trusting their LaTeX platform ***
% *** with production work. IEEE's font choices can trigger bugs that do  ***
% *** not appear when using other class files.                            ***
% The testflow support page is at:
% http://www.michaelshell.org/tex/testflow/



% Note that the a4paper option is mainly intended so that authors in
% countries using A4 can easily print to A4 and see how their papers will
% look in print - the typesetting of the document will not typically be
% affected with changes in paper size (but the bottom and side margins will).
% Use the testflow package mentioned above to verify correct handling of
% both paper sizes by the user's LaTeX system.
%
% Also note that the "draftcls" or "draftclsnofoot", not "draft", option
% should be used if it is desired that the figures are to be displayed in
% draft mode.
%
\documentclass[conference]{IEEEtran}
\usepackage[utf8]{inputenc}
% Add the compsoc option for Computer Society conferences.
%
% If IEEEtran.cls has not been installed into the LaTeX system files,
% manually specify the path to it like:
% \documentclass[conference]{../sty/IEEEtran}





% Some very useful LaTeX packages include:
% (uncomment the ones you want to load)


% *** MISC UTILITY PACKAGES ***
%
%\usepackage{ifpdf}
% Heiko Oberdiek's ifpdf.sty is very useful if you need conditional
% compilation based on whether the output is pdf or dvi.
% usage:
% \ifpdf
%   % pdf code
% \else
%   % dvi code
% \fi
% The latest version of ifpdf.sty can be obtained from:
% http://www.ctan.org/tex-archive/macros/latex/contrib/oberdiek/
% Also, note that IEEEtran.cls V1.7 and later provides a builtin
% \ifCLASSINFOpdf conditional that works the same way.
% When switching from latex to pdflatex and vice-versa, the compiler may
% have to be run twice to clear warning/error messages.






% *** CITATION PACKAGES ***
%
%\usepackage{cite}
% cite.sty was written by Donald Arseneau
% V1.6 and later of IEEEtran pre-defines the format of the cite.sty package
% \cite{} output to follow that of IEEE. Loading the cite package will
% result in citation numbers being automatically sorted and properly
% "compressed/ranged". e.g., [1], [9], [2], [7], [5], [6] without using
% cite.sty will become [1], [2], [5]--[7], [9] using cite.sty. cite.sty's
% \cite will automatically add leading space, if needed. Use cite.sty's
% noadjust option (cite.sty V3.8 and later) if you want to turn this off
% such as if a citation ever needs to be enclosed in parenthesis.
% cite.sty is already installed on most LaTeX systems. Be sure and use
% version 4.0 (2003-05-27) and later if using hyperref.sty. cite.sty does
% not currently provide for hyperlinked citations.
% The latest version can be obtained at:
% http://www.ctan.org/tex-archive/macros/latex/contrib/cite/
% The documentation is contained in the cite.sty file itself.






% *** GRAPHICS RELATED PACKAGES ***
%
\ifCLASSINFOpdf
  % \usepackage[pdftex]{graphicx}
  % declare the path(s) where your graphic files are
  % \graphicspath{{../pdf/}{../jpeg/}}
  % and their extensions so you won't have to specify these with
  % every instance of \includegraphics
  % \DeclareGraphicsExtensions{.pdf,.jpeg,.png}
\else
  % or other class option (dvipsone, dvipdf, if not using dvips). graphicx
  % will default to the driver specified in the system graphics.cfg if no
  % driver is specified.
  % \usepackage[dvips]{graphicx}
  % declare the path(s) where your graphic files are
  % \graphicspath{{../eps/}}
  % and their extensions so you won't have to specify these with
  % every instance of \includegraphics
  % \DeclareGraphicsExtensions{.eps}
\fi
% graphicx was written by David Carlisle and Sebastian Rahtz. It is
% required if you want graphics, photos, etc. graphicx.sty is already
% installed on most LaTeX systems. The latest version and documentation
% can be obtained at: 
% http://www.ctan.org/tex-archive/macros/latex/required/graphics/
% Another good source of documentation is "Using Imported Graphics in
% LaTeX2e" by Keith Reckdahl which can be found at:
% http://www.ctan.org/tex-archive/info/epslatex/
%
% latex, and pdflatex in dvi mode, support graphics in encapsulated
% postscript (.eps) format. pdflatex in pdf mode supports graphics
% in .pdf, .jpeg, .png and .mps (metapost) formats. Users should ensure
% that all non-photo figures use a vector format (.eps, .pdf, .mps) and
% not a bitmapped formats (.jpeg, .png). IEEE frowns on bitmapped formats
% which can result in "jaggedy"/blurry rendering of lines and letters as
% well as large increases in file sizes.
%
% You can find documentation about the pdfTeX application at:
% http://www.tug.org/applications/pdftex





% *** MATH PACKAGES ***
%
%\usepackage[cmex10]{amsmath}
% A popular package from the American Mathematical Society that provides
% many useful and powerful commands for dealing with mathematics. If using
% it, be sure to load this package with the cmex10 option to ensure that
% only type 1 fonts will utilized at all point sizes. Without this option,
% it is possible that some math symbols, particularly those within
% footnotes, will be rendered in bitmap form which will result in a
% document that can not be IEEE Xplore compliant!
%
% Also, note that the amsmath package sets \interdisplaylinepenalty to 10000
% thus preventing page breaks from occurring within multiline equations. Use:
%\interdisplaylinepenalty=2500
% after loading amsmath to restore such page breaks as IEEEtran.cls normally
% does. amsmath.sty is already installed on most LaTeX systems. The latest
% version and documentation can be obtained at:
% http://www.ctan.org/tex-archive/macros/latex/required/amslatex/math/





% *** SPECIALIZED LIST PACKAGES ***
%
%\usepackage{algorithmic}
% algorithmic.sty was written by Peter Williams and Rogerio Brito.
% This package provides an algorithmic environment fo describing algorithms.
% You can use the algorithmic environment in-text or within a figure
% environment to provide for a floating algorithm. Do NOT use the algorithm
% floating environment provided by algorithm.sty (by the same authors) or
% algorithm2e.sty (by Christophe Fiorio) as IEEE does not use dedicated
% algorithm float types and packages that provide these will not provide
% correct IEEE style captions. The latest version and documentation of
% algorithmic.sty can be obtained at:
% http://www.ctan.org/tex-archive/macros/latex/contrib/algorithms/
% There is also a support site at:
% http://algorithms.berlios.de/index.html
% Also of interest may be the (relatively newer and more customizable)
% algorithmicx.sty package by Szasz Janos:
% http://www.ctan.org/tex-archive/macros/latex/contrib/algorithmicx/




% *** ALIGNMENT PACKAGES ***
%
%\usepackage{array}
% Frank Mittelbach's and David Carlisle's array.sty patches and improves
% the standard LaTeX2e array and tabular environments to provide better
% appearance and additional user controls. As the default LaTeX2e table
% generation code is lacking to the point of almost being broken with
% respect to the quality of the end results, all users are strongly
% advised to use an enhanced (at the very least that provided by array.sty)
% set of table tools. array.sty is already installed on most systems. The
% latest version and documentation can be obtained at:
% http://www.ctan.org/tex-archive/macros/latex/required/tools/


% IEEEtran contains the IEEEeqnarray family of commands that can be used to
% generate multiline equations as well as matrices, tables, etc., of high
% quality.




% *** SUBFIGURE PACKAGES ***
%\ifCLASSOPTIONcompsoc
%  \usepackage[caption=false,font=normalsize,labelfont=sf,textfont=sf]{subfig}
%\else
%  \usepackage[caption=false,font=footnotesize]{subfig}
%\fi
% subfig.sty, written by Steven Douglas Cochran, is the modern replacement
% for subfigure.sty, the latter of which is no longer maintained and is
% incompatible with some LaTeX packages including fixltx2e. However,
% subfig.sty requires and automatically loads Axel Sommerfeldt's caption.sty
% which will override IEEEtran.cls' handling of captions and this will result
% in non-IEEE style figure/table captions. To prevent this problem, be sure
% and invoke subfig.sty's "caption=false" package option (available since
% subfig.sty version 1.3, 2005/06/28) as this is will preserve IEEEtran.cls
% handling of captions.
% Note that the Computer Society format requires a larger sans serif font
% than the serif footnote size font used in traditional IEEE formatting
% and thus the need to invoke different subfig.sty package options depending
% on whether compsoc mode has been enabled.
%
% The latest version and documentation of subfig.sty can be obtained at:
% http://www.ctan.org/tex-archive/macros/latex/contrib/subfig/




% *** FLOAT PACKAGES ***
%
%\usepackage{fixltx2e}
% fixltx2e, the successor to the earlier fix2col.sty, was written by
% Frank Mittelbach and David Carlisle. This package corrects a few problems
% in the LaTeX2e kernel, the most notable of which is that in current
% LaTeX2e releases, the ordering of single and double column floats is not
% guaranteed to be preserved. Thus, an unpatched LaTeX2e can allow a
% single column figure to be placed prior to an earlier double column
% figure. The latest version and documentation can be found at:
% http://www.ctan.org/tex-archive/macros/latex/base/


%\usepackage{stfloats}
% stfloats.sty was written by Sigitas Tolusis. This package gives LaTeX2e
% the ability to do double column floats at the bottom of the page as well
% as the top. (e.g., "\begin{figure*}[!b]" is not normally possible in
% LaTeX2e). It also provides a command:
%\fnbelowfloat
% to enable the placement of footnotes below bottom floats (the standard
% LaTeX2e kernel puts them above bottom floats). This is an invasive package
% which rewrites many portions of the LaTeX2e float routines. It may not work
% with other packages that modify the LaTeX2e float routines. The latest
% version and documentation can be obtained at:
% http://www.ctan.org/tex-archive/macros/latex/contrib/sttools/
% Do not use the stfloats baselinefloat ability as IEEE does not allow
% \baselineskip to stretch. Authors submitting work to the IEEE should note
% that IEEE rarely uses double column equations and that authors should try
% to avoid such use. Do not be tempted to use the cuted.sty or midfloat.sty
% packages (also by Sigitas Tolusis) as IEEE does not format its papers in
% such ways.
% Do not attempt to use stfloats with fixltx2e as they are incompatible.
% Instead, use Morten Hogholm'a dblfloatfix which combines the features
% of both fixltx2e and stfloats:
%
% \usepackage{dblfloatfix}
% The latest version can be found at:
% http://www.ctan.org/tex-archive/macros/latex/contrib/dblfloatfix/




% *** PDF, URL AND HYPERLINK PACKAGES ***
%
%\usepackage{url}
% url.sty was written by Donald Arseneau. It provides better support for
% handling and breaking URLs. url.sty is already installed on most LaTeX
% systems. The latest version and documentation can be obtained at:
% http://www.ctan.org/tex-archive/macros/latex/contrib/url/
% Basically, \url{my_url_here}.




% *** Do not adjust lengths that control margins, column widths, etc. ***
% *** Do not use packages that alter fonts (such as pslatex).         ***
% There should be no need to do such things with IEEEtran.cls V1.6 and later.
% (Unless specifically asked to do so by the journal or conference you plan
% to submit to, of course. )


% correct bad hyphenation here
\hyphenation{op-tical net-works semi-conduc-tor}


\begin{document}
%
% paper title
% can use linebreaks \\ within to get better formatting as desired
% Do not put math or special symbols in the title.
\title{\Large\bf Modelo educativo Coreano, riesgos y oportunidades}


% author names and affiliations
% use a multiple column layout for up to three different
% affiliations
\author{Rubén Darío Espinosa Roldán  \\
  Ingeniero de Sistemas\\
  Estudiante de maestría en ingeniería\\
  Universidad EAFIT  \\
  Medellín, Colombia}

% make the title area
\maketitle

% As a general rule, do not put math, special symbols or citations
% in the abstract
\begin{abstract}
\em
After 60 years of the war, Corea has become in one most of the higest economic powers at the globe, many studies suggest the change was possible basically for two reasons, industrial and education policies. Today Corea left the agricultural age behind (94\% in 1950 against 7\% today) and are living a industrial era, allowing a huge economy growth every year (12 richest country \cite{korea:rich}, how they have got to those goals based in a educational perspective and how can we apply those models in Colombia? 
\end{abstract}

% no keywords
\section*{Keywords}
\emph{
Korea, Colombia, Education, TIC's
}

\section{Introducción}
% no \IEEEPARstart
En este artículo se tratará el tema de la tecnología en la educación, tomando como ejemplo el caso Coreano, el cual es uno de los modelos educativos más existosos que existe en el mundo, ya que logro apalancar la economía del país en menos de 50 años, tienen una cobertura de educación cercana al 100\% en educación básica y media. Se buscara tambien hacer un paralelo con los esfuerzos que esta realizando Colombia en terminos de la incorporación y masificación de la tecnológia en la educación y algunos de los retos que enfrentará. 

\section{La Evolución}
Mientras en 1950 se disputaba la guerra con la cual conseguirían su independencia de Corea del norte, las condiciones educativas no eran las mejores a lo largo y ancho del país, las leyes incluían solamente la educación primaria, la infraestructura era casi nula, muchos de los estudiantes no tenían sillas y tomaban las clases en el piso, los profesores no tenían un grado de alfabetización alto y adicionalmente las familias no tenían los recursos económicos para tener educación privada.

A finales de los años 70 el control de la educación fue paulatinamente pasando de ser administrada por las localidades ó alcaldías para ser controladas desde el Ministerio de Educación Coreano, el cual empezó a tomar una serie de medidas las cuales debían ser aplicadas a nivel nacional, estas leyes incluían, entre otras, las siguientes medidas
\begin{itemize}
  \item Educación secundaria obligatoria
  \item Certificación de los educadores
  \item Asignación de recursos para el mejoramiento de las escuelas
  \item Rediseño de los textos escolares
\end{itemize}

En 1998 cuando ya se podía evidenciar resultados de las anteriores políticas aplicadas (el número de estudiantes matriculados con relación a la población en preescolar pasó de 1\% a 29\%, en primaria y secundaria pasó ser 70\% del 97\% \cite{moe}). se realizó una nueva reforma educativa, en la cual las TIC empiezan a ser tenidas en cuenta como base para la implementación de una educación más personalizada y basada en el seguimiento del desarrollo del alumno. Este plan se contempló en varias etapas:
\begin{itemize}
  \item Dotación de infraestructura en escuelas primarias y secundarias
  \item Creación de recursos educativos (contenidos)
  \item Dotación de equipos informaticos y manteniemiento de los mismos
\end{itemize}
Aunque el esfuerzo en la dotación de equipos informáticos ha sido importante, según estadísticas del Ministerio de Educación, hoy en día no se puede decir que hay un plan 1:1 totalmente implementado: 
\begin{center}
    \begin{tabular}{| l | l | l | l | l |}
    \hline
    Tipo & Num Estudiantes & Num PC & PC x Estudiante \\ \hline
    Primaria & 1,841,374 & 462,109 & 4.0\\ \hline
    Secundaria & 2,063,159 & 340,036 & 6.1\\\hline
    \end{tabular}
\end{center}
\emph{**El número de estudiantes incluye el total de profesores\cite{moe:one}}

\subsection{Grandes logros}
\begin{itemize}
  \item El cambio del modelo educativo ha contribuido enormemente en el avance cultural, económico y social de Corea, hoy en día ingresar a la universidad es un símbolo de estatus social, este es uno de los motivos por los que las familias invierten gran parte de sus ingresos en la educación básica y secundaria de sus hijos, la cual es complementada con clases particulares (\emph{En promedio un estudiante asiste a 50 horas semanales de clases, 10 horas más que el promedio mundial}). Esto ha llevado que el 90\% de los bachilleres ingresen a la universidad.
  \item La política industrial fue enfocada en sectores muy específicos lo cual permitía que las universidades se enfocáran en formar profesionales de excelente calidad y desarrollar programas de investigación en los temas que el mercadoestásolicitando, permitiendo así un crecimiento bidireccional y logrando un desarrollo económico importante.
  \item Se demuestra que una inversión del gobierno en educación es rentable desde todos los puntos de vista, ya que puede pasar de ser un país productor de materia prima a ser un productor de bienes y servicios el cual es el sector que más dinero genera.
\end{itemize}


\subsection{Grandes riesgos}
\begin{itemize}
\item Aunque no se puede culpar directamente al modelo educativo adoptado, la presión cultural y familiar que hay sobre los estudiantes para obtener buenos resultados académicos ha llevado que el 50\% de los estudiantes manifieste no ser feliz, un 16\% dice sentirse muy solo, en 2009 hubo más de 200 suicidios\cite{infelices}
\item Corea cambió la forma en que sus habitantes obtenían el sustento, dejaron atras la agricultura, pero se debe tener en cuenta que comparado con Colombia, hay una gran brecha en la disponibilidad de recursos naturales, en nuestro caso no podemos dejar el agro nunca, se debe buscar la manera de potenciarlo por medio de la tecnología, deben tener mayor apoyo de las universidades para que no sólo las industrias de las grandes ciudades sean beneficiadas por los avances en tecnología. 
\end{itemize}


\section{Colombia y las TIC}
El ministerio de educación Colombiano en los últimos años ha venido implementando una serie de programas en busca de una mejora en la calidad educativa de las instituciones educativas, estos planes contemplan la profesionalización de los docentes, la mejora de infraestructura física de las instituciones, creación de nuevas instituciones y la dotación tecnológica de las mismas.

En conjunto con el ministerio TIC se ha realizado un esfuerzo por disminuir la brecha en los temas de acceso a los recursos tecnológicos (Red, equipos de cómputo), con unas proyecciones a 2014 muy ambiciosas (1078 municipios con internet, el 96\% de ellos con fibra óptica, 50\% de los hogares con conexión a internet)\cite{mintic:red}, se han entregado más de 440.000 equipos de cómputo (355.000 computadores y 82.000 tabletas)\cite{mintic:hard} en instituciones educativas del país y se proyecta que para 2014 casi se duplicarán estas cifras. En temas de capacitación en el uso de las TIC se han capacitado 406.000 (1\% de la población) personas en el uso de internet  y la tecnología, de ellos 69.000 son docentes y directivos de instituciones educativas\cite{mintic:usuarios}.

La gran inversión que se viene realizando en infraestructura ha permitido que algunos actores del sistema educativo (Instituciones educativas, Alcaldías) empiecen a diseñar planes de mejoramiento de la calidad apalancados en los recursos disponibles, en caso de Medellín, algunas instituciones están implementando un sistema que permite a los estudiantes acceder a recursos educativos desde sus casas por medio de una aplicación de escritorios remotos llamada Citrix\cite{citrix}, sin embargo no hay una política nacional clara, la cual defina los parámetros o lineamientos que se deben seguir para el desarrollo de los mismos, por este motivo los riesgos de que estas iniciativas no prosperen en un futuro inmediato es muy alto dado que se requiere unas políticas bien definidas en los siguientes temas
\subsection{Cambio en el modelo educativo} 
El internet y la web 2.0 han cambiado el paradigma de la creación de contenidos, por lo que un estudiante puede ser un actor activo en el proceso de formación propio, por lo que se debería realizar cambios en pro de que el estudiante pueda generar, procesar, análizar e investigar en los temas de los clases. 

\subsection{Profesionalización de los educadores} 
Los educadores deben estar en constante actualización de sus conocimientos, para así poder incluir de una forma asertiva las tecnologías es la metodología de enseñanza.

\subsection{Desarrollo de contenido}
Se deben realizar contenidos que sean accesibles, verificables y de buena calidad que estén disponibles para toda la comunidad académica, ya que hoy en día en internet hay mucha información pero no toda es de calidad y no está respaldada por un ente que certifique la calidad y veracidad de la información.

\subsection{Mantenimiento de la infraestuctura}
Los esfuerzos de dotación de las instituciones no termina en la disposición inicial de los equipos, se debe manejar un programa de mantenimiento de equipos dañados y actualización de los mismos, ya que de nada sirve si en 10 años las instituciones continúan con los mismos recursos inicales, ya que quedarían obsoletos.

\subsection{Acompañamiento de la industria y universidades}   
Las instituciones educativas no son una parte aislada, se debe tener en cuenta que allí se están formando los futuros universitarios y trabajadores, por lo que si se genera una dinámica de colaboración entre estos se pueden lograr mejores resultados, el ejemplo de Corea nos muestra que cuando las universidades se involucran en la escuela, estos estudiantes se prepararán mejor para ingresar a la universidad y tendrán un mayor sentido de pertenencia hacia la misma. 


En este momento en Colombia, el municipio de itagui se encuenta en proceso de mejora de la educaciónen convenio con la Universidad EAFIT buscando la incorporación de tecnológia en las escuelas\cite{teso}, este proyecto es uno de los más ambiciosos que se han llevado en el país, ya que interviene en todos los actores de la educación, (Docentes, Estudiantes, Directivos, Padres de familia y Comunidad).

\subsubsection{Datos TESO}
\begin{itemize}
\item Se esta implementando un plan de computadores 1:1 con equipos XO
\item Un grupo de 30 educadores estan realizando una Maestría en Ingeniería como proceso de profesionalización
\item Se estan realizando talleres con los directivos de las instituciones
\item Se realizan laboratorios con estudiantes y profesores con el fin de evidenciar las actividades académicas que se pueden realizar con los equipos XO

\end{itemize}



% An example of a floating figure using the graphicx package.
% Note that \label must occur AFTER (or within) \caption.
% For figures, \caption should occur after the \includegraphics.
% Note that IEEEtran v1.7 and later has special internal code that
% is designed to preserve the operation of \label within \caption
% even when the captionsoff option is in effect. However, because
% of issues like this, it may be the safest practice to put all your
% \label just after \caption rather than within \caption{}.
%
% Reminder: the "draftcls" or "draftclsnofoot", not "draft", class
% option should be used if it is desired that the figures are to be
% displayed while in draft mode.
%
%\begin{figure}[!t]
%\centering
%\includegraphics[width=2.5in]{myfigure}
% where an .eps filename suffix will be assumed under latex, 
% and a .pdf suffix will be assumed for pdflatex; or what has been declared
% via \DeclareGraphicsExtensions.
%\caption{Simulation Results.}
%\label{fig_sim}
%\end{figure}

% Note that IEEE typically puts floats only at the top, even when this
% results in a large percentage of a column being occupied by floats.


% An example of a double column floating figure using two subfigures.
% (The subfig.sty package must be loaded for this to work.)
% The subfigure \label commands are set within each subfloat command,
% and the \label for the overall figure must come after \caption.
% \hfil is used as a separator to get equal spacing.
% Watch out that the combined width of all the subfigures on a 
% line do not exceed the text width or a line break will occur.
%
%\begin{figure*}[!t]
%\centering
%\subfloat[Case I]{\includegraphics[width=2.5in]{box}%
%\label{fig_first_case}}
%\hfil
%\subfloat[Case II]{\includegraphics[width=2.5in]{box}%
%\label{fig_second_case}}
%\caption{Simulation results.}
%\label{fig_sim}
%\end{figure*}
%
% Note that often IEEE papers with subfigures do not employ subfigure
% captions (using the optional argument to \subfloat[]), but instead will
% reference/describe all of them (a), (b), etc., within the main caption.


% An example of a floating table. Note that, for IEEE style tables, the 
% \caption command should come BEFORE the table. Table text will default to
% \footnotesize as IEEE normally uses this smaller font for tables.
% The \label must come after \caption as always.
%
%\begin{table}[!t]
%% increase table row spacing, adjust to taste
%\renewcommand{\arraystretch}{1.3}
% if using array.sty, it might be a good idea to tweak the value of
% \extrarowheight as needed to properly center the text within the cells
%\caption{An Example of a Table}
%\label{table_example}
%\centering
%% Some packages, such as MDW tools, offer better commands for making tables
%% than the plain LaTeX2e tabular which is used here.
%\begin{tabular}{|c||c|}
%\hline
%One & Two\\
%\hline
%Three & Four\\
%\hline
%\end{tabular}
%\end{table}


% Note that IEEE does not put floats in the very first column - or typically
% anywhere on the first page for that matter. Also, in-text middle ("here")
% positioning is not used. Most IEEE journals/conferences use top floats
% exclusively. Note that, LaTeX2e, unlike IEEE journals/conferences, places
% footnotes above bottom floats. This can be corrected via the \fnbelowfloat
% command of the stfloats package.



\section{Conclusiones}
Todos los planes de mejoramiento de la educación deben ser contemplados a un largo plazo, ya que se requiere un cambio generacional para poder evidenciar los resultados, cualquier plan a de cambio a corto plazo va a arrojar resultados insignificantes, por esto, deben ser liderados como una política de estado con el fin de asegurar los recursos económicos y la estabilidad de los procesos.

El papel del docente no podra ser reemplazado por la tecnólogia, sin embargo el papel del profesor pasa a ser al de un acompañante en el proceso cognitivo del estudiante, el cual lo guia segun el propio avance del estudiante.

Una de las grandes quejas de las empresas hoy en día, es que los estudiantes no llegan bien preparados para ingresar al mundo laborar, pero sin embargo estas se niegan a trabajar en conjunto con las universidades, es {\it \bf necesario} que las empresas participen en grupos de investigación en la universidades, así crear un vinculo entre sus necesidades y como se deben preparar los estudiantes.




% trigger a \newpage just before the given reference
% number - used to balance the columns on the last page
% adjust value as needed - may need to be readjusted if
% the document is modified later
%\IEEEtriggeratref{8}
% The "triggered" command can be changed if desired:
%\IEEEtriggercmd{\enlargethispage{-5in}}

% references section

\begin{thebibliography}{1}
  
\bibitem{korea:rich}
Wikipedia, List of countries by GDP,
http://en.wikipedia.org/wiki/List\_of\_countries\_by\_GDP\_(PPP)  

\bibitem{moe}
Ministery Of Education, \emph{Primary and Secondary Education}
http://english.moe.go.kr/web/1729/site/contents/en/en\_0224.jsp

\bibitem{moe:one}
Ministery Of Education, \emph{Primary and Secondary Education, Computers in schools}
http://english.moe.go.kr/web/1729/site/contents/en/en\_0227.jsp

\bibitem{politicas}
Gonzales Pérez, Alicia, \emph{Políticas Educativas en Corea del Sur: buenas prácticas TIC en la sociedad del conocimiento} 
http://www.academia.edu/520177/Politicas\_educativas\_en\_Corea\_del\_Sur
\_Buenas\_practicas\_TIC\_en\_la\_sociedad\_del\_conocimiento

\bibitem{infelices}
Trahtember, Leon, Diario de América 2011, \emph{Corea del Sur: alumnos infelices pero con buenas notas}
http://www.trahtemberg.com/articulos/1700-corea-del-sur-alumnos-infelices-pero-con-buenas-notas.html

\bibitem{mintic:red}
Ecosistema Digital, Infraestructura,
http://www.vivedigital.gov.co/participa/
ecosistema-digital/

\bibitem{mintic:hard}
Ecosistema Digital, Servicios,
http://www.vivedigital.gov.co/participa/
ecosistema-digital/

\bibitem{mintic:usuarios}
Ecosistema Digital, Usuarios,
http://www.vivedigital.gov.co/participa/
ecosistema-digital/

\bibitem{citrix}
Citrix, Workbetter,
https://www.youtube.com/watch?v=uef-G\_0cKSs
\&list=PLbrnMEdPVDPWOkxGjSrKn6G-4iJoFti2f

\bibitem{teso}
Plan Digital TESO,
http://www.planteso.edu.co/site/index.php/el-plan/instituciones


\end{thebibliography}




% that's all folks
\end{document}


